\section{Arquitecturas}

En la medida que se intenta mantener una realidad compartida entre cada vez más clientes, la segunda parte del \emph{state melding}, la diseminación de las actualizaciones se hace cada vez más importante debido a las limitaciones en la capacidad de procesamiento y la red. A continuación se nombran algunos modelos de arquitecturas que enfrentan este problema de escalabilidad. Todo son mencionados en \cite{liu2012survey}, salvo \emph{mirrored game architecture} que se menciona en \cite{cronin2004efficient}.
% TODO: Hay más referencias donde se desarrolla más mirrored game architecture, buscarlas.

\subsection{Detalles de las arquitecturas}

\subsubsection{Simulator-centric}

Utilizado por \emph{Second-Life} y \emph{Eve Online} (un juego de temática espacial), en esta arquitectura un servidor tiene autoridad sobre un área del mundo central que impone un órden sobre los eventos y mantiene una copia autoritaria de la simulación. Esta solución centralizada obliga a la separación del mundo virtual en regiones espaciales, en el caso de \emph{Second-Life} se trata de regiones cuadradas, cada una es manejada por sólo un proceso simulador, el que se encarga de diseminar las actualizaciones y aplicar todas las operaciones a los objetos; mientras que en el caso de \emph{Eve Online} cada sistema solar está controlado por un proceso en un servidor, llamado servidor \emph{SOL}.

En \emph{Second-Life} hay \emph{sequential consistency} (por definición, ya que no hay estados compartidos), adicionalmente, cuando los clientes tienen autorización para modificar un objeto, este objeto se \emph{congela} en el simulador, evitando dobles escrituras. Los mensajes se ejecutan al momento que los recibe el servidor \footnote{Este enfoque es muy parecido al protocolo basado en dead-reckoning implementado.}. El simulador sólo envía a los clientes los objetos de su área de visión y estos aplican \emph{dead-reckoning}. En \emph{Eve Online} las actualizaciones mandan las condiciones iniciales del estado para un periodo de tiempo, pudiendo los clientes revertir cambios para ajustarse a esas condiciones y detectar cómo eso altera el estado actual.

\emph{Eve Online} tiene algunas características adicionales, en primer lugar los clientes no se conectan directamente a los servidores SOL, sino que se conectan a un servidor \emph{proxy}, el que a su vez se conceta através de una VPN con un SOL, esto disminuye la carga de los servidores SOL. Adicionalmente, utiliza burbujas de consecuencialidad, preocupándose sólo de hacer secuenciales las acciones de grupos de jugadores que estén suficientemente cerca como para interactuar, cada cliente aplica \emph{dead-reckoning} con los objetos que se encuentran en la misma burbuja de consecuencialidad.

\subsubsection{Darkstar}

\emph{Darkstar} un intento por crear una infraestructura para mundos virtuales del lado del servidor orientada a diferentes tipos de juego. Busca evita el cuello de botella de las arquitecturas \emph{simulator-centric} separando la data de su procesamiento, de manera que este último se pueda realizar en \emph{hardware} designado dínamicamente. El estado del juego entonces se mantiene en servidores de bases de datos que son modificados por tareas que corren en \emph{game servers} que son activadas por eventos de los clientes.

Al igual que un \emph{simulator-centric} cada \emph{game server} se encarga de mantener la \emph{serializability} imponiendo un órden a los eventos de los clientes, y manteniendo una copia autoritaria (para los clientes conectados) del estado, esta copia al mismo tiempo es un \emph{cache} de lo almacenado en la base de datos. Las modificaciones en \emph{cache} se realizan desde los \emph{game servers} mediante transacciones siendo este otro nivel de serialización. Una meta de diseño de este proyecto era que los clientes interesados en cierta data se puedieran conectar a \emph{game servers} que ya la estuvieran manejando, para balancear mejor la carga y evitar inconsistencias mayores.

Para que este esquema sea efectivo se requiere de una base de datos distribuída de alto-rendimiento que soporte transacciones \emph{ACID}, este proyecto usa \emph{Berkeley DB}. Aun así, este proyecto fue descontinuado, y aunque se hizo un fork de la comunidad, llamado \emph{RedDrawf}, no se encontraron evidencias de actividad más recientes que del 2013, lo que puede deberse a un bajo desempeño en la práctica, pues autores como \cite{fujita2006new} señalan que la disponibilidad global de un objeto y su acceso aleatorio eficiente para videojuegos altamente interactivos podrían no ser realisticamente posibles.

\subsubsection{Croquet}

Este modelo presentado en el proyecto Croquet, usa una arquitectura \emph{p2p} descentralizada separando los objetos en islas de las que cada cliente tiene una replica local. Adicionalmente, y para lograr \emph{sequential consistency}, los eventos pasan por un nodo \emph{router} centralizado que se encarga simplemente de colocar marcas de tiempo a los eventos.

Croquet reduce el tráfico de la red usando una vista orientada a los objetos, cada participante simula el comportamiento de los objetos que \emph{hostea}, las variables internas del objeto están encapsuladas y funcionan en base a comportamientos activados por eventos (más ligeros que los estados internos), que pueden recibir de otros objetos.

\emph{Croquet} dejó de desarrollarse para dar paso a \emph{Open Cobalt}, una plataforma para construír y compartir portales 3D virtuales sin la necesidad de servidores centralizados. Ambos proyectos usan una versión modificada de un protocolo basado en \emph{peers} llamado \emph{TeaTime}.

\subsubsection{Mirrored game architecture}

% Comentado por Cronin, Kurc, Filstrup y Jamin (for mirrored game architectures).

% Meta arquitecturas
% - interest management, se envian actualizaciones relevantes a cada cliente, menos relevantes (e.g. objetos más lejanos) se envían con menos frecuencia. Puede ser spatial based o class based.

% El uso de islas o burbujas de consecuencialidad como una manera de lograr una mayor escalabilidad.

% El enfoque de Croquet se ve bueno.
