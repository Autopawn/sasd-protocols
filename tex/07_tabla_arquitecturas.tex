\section{Comparación de arquitecturas}

% Tabla comparativa de las arquitecturas, sobre:
% - Consistencia y responsividad
% - Seguridad % TODO: Ponerlo abajo.
% - Escalabilidad
%
% Clasificar según 5° paper del informe de avance.

\small
\begin{longtable}{|p{3cm}|p{6cm}|p{6cm}|}
\hline
\textbf{Arquitectura} & \textbf{Consistencia vs responsividad} & \textbf{Escalabilidad} \\ \hline
Simulator-Centric &
El \emph{server} asegura \emph{sequential consistency}, sin embargo el que los eventos deban llegarle a este y que el mismo deba procesar la evolución de los estados disminuye la velocidad con que llegan las actualizaciones, sobre todo \emph{cliente-cliente}. Técnicas como \emph{dead-reckoning} se usan para simular una mejor \emph{responsividad}. &
El servidor se transforma en un cuello de botella, tanto por la capacidad de procesamiento como de la red, aun cuando esto se puede mejorar usando \emph{proxies}. Generalmente se tiene que optar por construír con partes aisladas y limitar la cantidad de jugadores a dichas partes. \\
\hline
Datos separados de computación (Darkstar) &
Antes de que se pueda asegurar consistencia debe transcurrir mucho tiempo, lo que da paso a mucha inconsistencia (sobretodo si clientes se conectan a servidores distintos que acceden a la misma data) o una responsividad muy baja. &
Esta arquitectura evita la necesidad de crear \emph{islas} de objetos y se puede escalar horizontalmente de manera transparente, esto si se logran superar los problemas de consistencia y responsividad. \\
\hline
Peer-to-peer con \emph{router} (Croquet) &
La responsividad puede ser incluso mejor que \emph{simulator-centric} ya que no es necesario hacer procesamiento de estados en el \emph{router} para informar que se realizaron los eventos, más aun, los \emph{clientes} pueden enviar sus eventos directamente con una estimación del tiempo de ejecución que recibirán finalmente, que en caso de que sean diferentes se pueden corregir después. &
El \emph{router} no necesita mantener una copia de la simulación lo que requiere menos procesamiento. \newline
Enviar eventos en vez de objetos evita congestionar la red. \newline
Si las \emph{islas} de objetos se pueden organizar de manera efectiva, no debería ser significativa la carga de los clientes por replicar el estado, de otra manera y a gran escala tanto este beneficio como los anteriores pueden verse afectados. \newline El sólo envío de eventos requiere mecanismos de \emph{rollback} que hacen un poco más pesada la carga en el cliente. \\
\hline
Mirrored game architectures &
Desde el punto de vista del cliente el \emph{retraso} debería ser igual que en un \emph{simulator-centric}, sin embargo, si se tienen dos clientes conectados a \emph{mirrors} diferentes, verán un retraso mayor entre ellos (dado que los \emph{mirrors} deben coordinarse), por lo que tiene más sentido utilizarlo para juegos coolaborativos.  &
Sin una técnica que distribuya los estados entre los \emph{mirrors}, cada \emph{mirror} tiene que tener una copia completa del estado y realizar todo el procesamiento, por lo que parece ser una mejor opción que \emph{simulator-centric} sólo cuando el cuello de botella está en la cantidad de jugadores que pueden conectarse o, más generalmente, en la capacidad de la red.\\
\hline
\end{longtable}
\normalsize

% El \emph{router} asegura tiempos de ejecución correctos para los eventos de los clientes.
% NOTE: Señalar que aquí se pondrán conclusiones también en la introducción.

% TODO: Seguridad respecto a caídas del sistema.
