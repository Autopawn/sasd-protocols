\section{Comparación de arquitecturas}

% Tabla comparativa de las arquitecturas, sobre:
% - Consistencia y responsividad
% - Seguridad
% - Escalabilidad
%
% Clasificar según 5° paper del informe de avance.

\small
\begin{longtable}{|p{2cm}|p{4cm}|p{4cm}|p{4cm}|}
\hline
\textbf{Arquitectura} & \textbf{Consistencia vs responsividad} & \textbf{Seguridad} & \textbf{Escalabilidad} \\ \hline
Simulator-Centric &
consistencia y responsividad &
seguridad &
El servidor se transforma en un cuello de botella, tanto por la capacidad de procesamiento como de la red, aun cuando esto se puede mejorar usando \emph{proxies}. Generalmente se tiene que optar por construír con partes aisladas y limitar la cantidad de jugadores a dichas partes. \\
\hline
Datos separados de computación (Darkstar) &
consistencia y responsividad &
seguridad &
escalabilidad \\
\hline
Peer-to-peer con \emph{router} (Croquet) &
consistencia y responsividad &
El \emph{router} asegura tiempos de ejecución correctos para los eventos de los clientes. &
El \emph{router} no necesita mantener una copia de la simulación lo que requiere menos procesamiento. \newline
Enviar eventos en vez de objetos evita congestionar la red. \newline
Si las \emph{islas} de objetos se pueden organizar de manera efectiva, no debería ser significativa la carga de los clientes por replicar el estado, de otra manera y a gran escala tanto este beneficio como los anteriores pueden verse afectados. \\
\hline
Mirrored game architectures &
consistencia y responsividad &
seguridad &
escalabilidad \\
\hline
\end{longtable}
\normalsize

% NOTE: Señalar que aquí se pondrán conclusiones también en la introducción.
