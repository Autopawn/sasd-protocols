\section{Comparación de protocolos}

% Tabla comparativa de protocolos, sobre:
% - Consistencia y responsividad
% - Seguridad
% - Escalabilidad
%
% Clasificar según 3° paper del informe de avance.

\small
\begin{longtable}{|p{2cm}|p{4cm}|p{4cm}|p{4cm}|}
\hline
\textbf{Protocolo}
&
\textbf{Consistencia vs responsividad}
&
\textbf{Seguridad}
&
\textbf{Escalabilidad}
\\ \hline
Centralizado con \emph{dead-reckoning}
&
El retraso aumenta dado que los eventos tienen que ser procesados por el servidor, se puede aplicar un \emph{local-lag} una vez los eventos llegan al servidor para disminuír las inconsistencias en los clientes, pero a costa de una responsividad aun peor.
&
Que los eventos se ejecuten al momento que lleguen al servidor y que este mantenga una copia autoritaria del estado del juego asegura que los clientes no abusen con los tiempos de ejecución asociados a sus eventos.
&
El servidor se convierte en un cuello de botella, por el costo de procesamiento de correr una instancia del juego y por la capacidad de la red, ya que debe conectarse con todos los clientes.
\\ \hline
Time-warp
&
Los \emph{local-lags} se pueden ajustar para balancear de la mejor manera posible los efectos del retraso de la red entre consistencia y responsividad.
&
Por si sólo, no hay forma de asegurar que los clientes asocien tiempos de ejecución correctos a sus eventos, pudiendo lograr una mayor responsibidad afectando la consistencia del resto.
&
A parte de tener que hacer la actualización de estado determinista, abstrae la programación del juego. \newline
Se puede extender a varios clientes sin modificar el código, pero por ser \emph{p2p} está limitado por la cantidad de conexiónes, requiriendo un método para estructurar la red más complejo que conectar todos con todos. El mantener y copiar constantemente estados anteriores y los \emph{rollbacks} requieren verias veces más capacidad de procesamiento y memoria por parte de los clientes.
\\ \hline
Fast event ordering
&
Los retrasos de la red afectan completamente responsividad en pos de lograr consistencia total.
&
Conociendo cómo funciona el algoritmo, un cliente puede mentir respecto a sus retrasos con los otros clientes para beneficiarse o afectar el correcto funcionamiento de la red, al igual que los otros protocolos \emph{p2p} puede enviar eventos diferentes a clientes diferentes para hacer diverger las replicas.
&
Al igual que \emph{time-warp} está limitado por la cantidad de conexiones, si se busca una mayor cantidad de clientes y que estos no esté conectados todos con todos se requiere un método para hacer una red estructurada.
\newline Consume muchos menos recursos de parte del cliente.
\\ \hline
Trailing State Syncronization
&
La corrección de inconsistencias no es inmediata como en \emph{timewarp}, se espera hasta que se detectan, esto se suma al retraso de la red, por otro lado, si los eventos no afectan el estado del resultado, no se requiere hacer \emph{rollback}.
&
Tiene los mismos problemas que \emph{time-warp}.
&
Que se espere hasta detectar la inconsistencia permite resolver varias inconsistencias al mismo tiempo, ahorrando procesamiento, al mismo tiempo, se requiere mucho menos memoria para guardar los estados anteriores, aunque se debe avanzar todos los \emph{trailing states} cada \emph{frame}.
\\ \hline
\end{longtable}
\normalsize

% TODO: Conclusiones (leer paper de seguridad).
% En general puede ser malo revelar todo el estado al jugador, si se ocultan cosas al nivel del displayer del cliente pueden hacerse injusticias...

% TODO: Se destaca que una característica deseable puede ser el que los clientes afectados por inconsistencias o mala responsibidad sean los que tienen peor red respecto al resto.

% TODO: Explicar un poco como cambiarán las cosas si se distribuyera el procesamiento de los estados.

% TODO: Señalar los protocolos p2p y sus problemas (enviar eventos diferentes a otros clientes para hacer diverger el estado del juego o enviar eventos en el pasado) el problema de llegar a un consenso o excluír al jugador con faltas.

% NOTE: Señalar que aquí se pondrán conclusiones también en la introducción.
