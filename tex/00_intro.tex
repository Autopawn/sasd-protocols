\section{Introducción}

El objetivo de este trabajo es realizar una comparación práctica de diferentes métodos para lograr consistencia de un estado compartido entre varios clientes en un videojuego, y, posteriormente, realizar una comparación teórica sobre las diferentes arquitecturas que soporten un videojuego masivo multijugador, evaluando la compatibilidad y desempeño de estas con los protocolos, y haciendo un análisis de cuál sería la experiencia del usuario, cómo se manejan las inconsistencias, su seguridad y escalabilidad.

Dentro de este análisis se consideran juegos que requieren rápidos tiempos de respuesta, como son los \emph{shooters}, en desmedro de los que no, por ejemplo, juegos de estrategia basados en turnos.

% TODO: Indicar separación del trabajo:
% en el capítulo 2 se presentarán los protocolos implementados...
% ^ en el capítulo 4 se explicará más en detalle cómo se implementaron los protocolos y se simularon los problemas de conexión o redes.

% TODO: El trabajo debe enfocarse más en la combinación de las arquitecturas y los protocolos, sobretodo en las conclusiones.

% Señalar el interés: no sólo en esta área pero en varias otras, (sistemas distribuídos, bases de datos distribuídas, sistemas de edición colaborativa) pero la diferencia crucial es la presencia de eventos continuos (haciendo obsoletos los otros modelos de consistencia).
