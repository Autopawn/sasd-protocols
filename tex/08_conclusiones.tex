\section{Conclusiones}

% Concluír cosas...
%
% E.g.:
% - Qué protocolos son más convenientes para qué arquitecturas.
% - Qué protocolos y arquitecturas son mejores para qué tipo de juego (considerar por ejempo la cantidad de jugadores, la reactividad necesaria, cómputo necesario, consistencia, etc.).
% - Qué tecnologías se pueden utilizar para prevenir problemas de seguridad de los protocolos que los tienen.
% - ¿Algunas conclusiones sobre descentralización?
% - Qué cosas se podrían investigar más.
% - ... ¡Y muchas más!

% Hay una diferencia entre hacer dead reckoning con objetos manejados por cada usuario a hacerlo con eventos... elaborate more.

% Local lag en el cliente, local lag en el servidor, servidor resuelve el problema de seguridad de que los clientes pueden mentir respecto al local lag.

% Eventos "suaves" y eventos "duros", ¿En qué paper leí eso? ¿El de TSS?

% Aunque el trabajo original indica que se puede usar \emph{dead-reckoning} para hacer aparentemente más responsivo la simulación, esto no es posible sin lograr inconsistencias a nivel local, por lo que se requeriría, por ejemplo, algún método de interpolación para \emph{suavizar} la percepción de estas inconsistencias -> ojo, pero esto rompería el que se cumplan las reglas físicas en ciertos casos de mucha discrepancia. señalar que \cite{li2004supporting} hace esto (aunque no se programó).

% TODO: Agregar y revisitar conclusiones del informe de avance.

% El que la actualización de los estados respecto al tiempo sea determinsta es otra forma de evitar la necesidad de actualizaciones constantes y permitir otros modelos de consistencia, ¿Se convierte en una necesidad? <- Resolver esto.
% El problema es la lectura constante de la pantalla por parte del usuario final.

% Fenómenos de seguridad que pasan ante un usuario con percepción limitada, cómo se puede aprovechar.

% Importancia de los cambios "importantes", no se trató, el identificarlos, delay de confirmación, muchas otras cosas... (separar cambios importantes de cambios suceptibles a reparación).

% Temas de justicia, ¿Penalizar peores conexiones a la red?

% En los juegos coolaborativos se puede tolerar más las inconsistencias.

% Sistema híbrido no determinista en que haya comunicación p2p y un servidor con estados mandatorios para coordinar cada cierto tiempo.

% Arquitecturas más avanzadas combinan los métodos y protocolos, categorizando eventos según sus propiedades.
