\section{Conclusión}

% Concluír cosas...
%
% E.g.:
% - Qué protocolos son más convenientes para qué arquitecturas.
% - Qué protocolos y arquitecturas son mejores para qué tipo de juego (considerar por ejempo la cantidad de jugadores, la reactividad necesaria, cómputo necesario, consistencia, etc.).
% - Qué tecnologías se pueden utilizar para prevenir problemas de seguridad de los protocolos que los tienen.
% - ¿Algunas conclusiones sobre descentralización?
% - Qué cosas se podrían investigar más.
% - ... ¡Y muchas más!

% Hay una diferencia entre hacer dead reckoning con objetos manejados por cada usuario a hacerlo con eventos... elaborate more.

% Local lag en el cliente, local lag en el servidor, servidor resuelve el problema de seguridad de que los clientes pueden mentir respecto al local lag.

% Eventos "suaves" y eventos "duros", ¿En qué paper leí eso? ¿El de TSS?

% Aunque el trabajo original indica que se puede usar \emph{dead-reckoning} para hacer aparentemente más responsivo la simulación, esto no es posible sin lograr inconsistencias a nivel local, por lo que se requeriría, por ejemplo, algún método de interpolación para \emph{suavizar} la percepción de estas inconsistencias.

% TODO: Agregar y revisitar conclusiones del informe de avance.
