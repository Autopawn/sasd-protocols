\section{Implementación}


\subsection{Estructura general del \emph{framework}}

% 

\subsection{Simulación de la latencia entre clientes}


% Descripción del juego de pruebas, extraer parte importante del informe de avance.

% - Implementación de los protocolos y cómo se diseñó un sistema general para tratar con todos ellos.
% - Implementación de las simulaciones de latencia, pérdida de paquetes y jitter.
% - Setup de los nodos en la red.

% Señalar que se requerián todos los clientes conectados al principio para simplificar.

% Señalar que vamos a abstraer a los clientes de lo que hace el servidor, salvo por el hecho de que el cliente sabe que a través del servidor se hará el broadcast.

% La programación del juego está dada por una función determinista f(s_i,e) = s_{i+1}, donde s_i es el estado en una \emph{frame} específica y e es un conjunto de eventos realizados por usuarios.
