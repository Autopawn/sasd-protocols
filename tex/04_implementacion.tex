\section{Implementación}

\subsection{Videojuego}

El videojuego consiste en una serie de cuadrados \emph{negras} y de colores, las primeras se mueven automáticamente producto de reglas físicas previamente determinadas, mientras que las segundas son manejadas por los diferentes clientes (mediante el control de su aceleración), se tienen las siguiente reglas:

\begin{itemize}
\item Si dos pelotas de colores colisionan, la que lleva menos velocidad se destruye (luego vuelve a aparecer en otro lugar del mapa) y el jugador que maneja la otra gana un punto.
\item Si en una colisión al menos una de las pelotas es negra, ambas rebotan.
\item Si una pelota sale del mapa, es destruida (volviendo a aparecer en otro lado), si es de color el jugador que la maneja pierde un punto.
\end{itemize}

\subsection{Estructura general del \emph{framework}}

La implementación se divide en una sección de manipulación de \emph{streams} por la red, otra que actualiza los estados de la aplicación en respuesta a eventos y una tercera que se encarga de renderizar el estado de la aplicación.

Esta elección de estructura se basa en la modularidad de cada funcionalidad, ya que el manejo de la red se puede reutilizar para distintos protocolos y la lógica del cliente es completamente atómica y \textbf{determinista}.

El determinismo es clave para la implementación del \emph{framework}, ya que la sincronización no se basa a partir de paquetes que definen el estado en los distintos clientes, si no que los eventos emitidos en cada uno de ellos. De esta manera, si la lógica de la aplicación es determinista, basta con intercambiar el input percibido en cada cliente con el resto de los nodos para poder implementar la sincronización dependiendo del protocolo escogido.

Como detalles adicionales, cabe destacar el desarrollo de diferentes utilidades que ofrecen un uso más sencillo de estructuras de datos dinámicas y serialización de datos. Además en la sección de renderización del cliente, el dibujo de cada iteración no es borrado completamente de la pantalla para poder ver claramente la consistencia que otorga el protocolo.\footnote{https://github.com/Autopawn/sasd-protocols}

\subsection{Simulación de la latencia entre clientes}

Para hacer comparaciones entre las distintas implementaciones de protocolos se desarrolla un método de simulación de latencia entre clientes. Para mantener la simulación simple, cada cliente es inicializado con una matriz de latencias con dimensiones $n\times n$ donde $n$ es la cantidad máxima de jugadores. Como su nombre lo dice, esta matriz relaciona la latencia \textbf{dirigida} entre 2 jugadores.

Para comprender su funcionamiento, se deben entender los conceptos:
\begin{itemize}
	\item \emph{frame}: se trata del número de iteración del procesamiento en el lado del cliente en el que cierto evento debe ser ejecutado.
	\item \emph{summoned\_frame}: se trata del número de iteración del procesamiento en el lado del cliente en el que cierto evento fue emitido.
\end{itemize}

Entonces, si no existe latencia simulada entre 2 clientes, un paquete es puesto en cola para ser procesado en la iteración \emph{frame} inmediatamente al ser recibido, mientras que, si existe una latencia simulada de \emph{lat} iteraciones, ese mismo paquete será retenido, y se pondrá en cola para la iteración \emph{frame} recién cuando el cliente esté en la iteración \emph{summoned\_frame} + \emph{lat}.

% Descripción del juego de pruebas, extraer parte importante del informe de avance.

% - Implementación de los protocolos y cómo se diseñó un sistema general para tratar con todos ellos.
% - Implementación de las simulaciones de latencia, pérdida de paquetes y jitter.
% - Setup de los nodos en la red.

% Señalar que se requerián todos los clientes conectados al principio para simplificar.

% Señalar que vamos a abstraer a los clientes de lo que hace el servidor, salvo por el hecho de que el cliente sabe que a través del servidor se hará el broadcast.

% La programación del juego está dada por una función determinista f(s_i,e) = s_{i+1}, donde s_i es el estado en una frame específica y e es un conjunto de eventos realizados por usuarios.

\subsection{Networking engine}
Para la implementación de las interacciones de red, se usa como base el protocolo \texttt{TCP}. Esta elección se sustenta en que no es necesario tomar en cuenta el orden en que se envían los mensajes y su integridad; si bien es cierto que existen desventajas respecto a la latencia por el \emph{overhead} que produce, no causan mayor impacto en un entorno local.

El mayor problema que supone el uso de \texttt{TCP} para nuestros propósitos, es que a nivel de usuario se trabaja como un stream de datos en vez de estar encapsulados en paquetes atómicos, debido a esto, el uso de este protocolo, supone mayor manipulación de la información para poder aislar los distintos mensajes enviados.

Para facilitar estas interacciones, se desarrolla un sistema personalizado que se compone de 2 partes principales:

\subsubsection*{Server-side}
Para mantener la implementación sencilla, el servidor se encarga de levantar un \emph{thread} por cada cliente conectado. Cada uno de ellos sigue la siguiente secuencia:

\begin{enumerate}
	\item Esperar a que haya un \emph{stream} de datos listo para leer o recibir una notificación indicando que el \emph{buffer} de envío (\emph{stream} en cola que se debe enviar al cliente conectado) ha sido escrito por otro \emph{thread}. \textbf{Nota:} ambas cosas pueden pasar simultáneamente.
	\item Recibir los datos desde el cliente remoto (si es que hay información disponible para leer) y agregarlos a un \emph{buffer} acumulativo.
	\item Si hay suficiente información en el \emph{buffer} acumulativo, seccionarlo tantas veces como sea posible en paquetes delimitados.
	\item Escribir todos los paquetes seccionados a un \emph{buffer} histórico. Este buffer mantiene el stream que ha enviado el cliente durante todo el tiempo que ha estado conectado.
	\item Escribir todos los paquetes seccionados en el \emph{buffer} de envío de todos los otros clientes conectados.
	\item Enviar toda la información acumulada en el \emph{buffer} de envío.
\end{enumerate}

% \item Recibir los datos desde el cliente remoto (leer datos recibidos) y agregarlos a un \emph{buffer} acumulativo.
% \end{enumerate}

Con estos sencillos pasos se puede lograr un sistema que sea capaz de admitir conexiones en distintos puntos en la línea temporal manteniendo a todos los clientes sincronizados.

\subsubsection*{Client-side}

Para que las operaciones de red no interfieran con la lógica del cliente (que actualiza y renderiza el estado del juego), se genera una librería que hace la tarea de seccionar un \emph{stream} entrante en paquetes del protocolo, en segundo plano, de modo que se pueda obtener una cola de paquetes recibidos asíncronamente en la fase de procesamiento de eventos.
