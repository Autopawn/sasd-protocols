\section{Implementación}

% Descripción del juego de pruebas, extraer parte importante del informe de avance.

% - Implementación de los protocolos y cómo se diseñó un sistema general para tratar con todos ellos.
% - Implementación de las simulaciones de latencia, pérdida de paquetes y jitter.
% - Setup de los nodos en la red.

% Señalar que se requerián todos los clientes conectados al principio para simplificar.

% Señalar que vamos a abstraer a los clientes de lo que hace el servidor, salvo por el hecho de que el cliente sabe que a través del servidor se hará el broadcast.

% La programación del juego está dada por una función determinista f(s_i,e) = s_{i+1}, donde s_i es el estado en una frame específica y e es un conjunto de eventos realizados por usuarios.

\subsection{Networking engine}
Para la implementación de las interacciones de red, se usa como base el protocolo \texttt{TCP}. Esta elección se sustenta en que no es necesario tomar en cuenta el orden en que se envían los mensajes y su integridad; si bien es cierto que existen desventajas respecto a la latencia por el \emph{overhead} que produce, no causan mayor impacto en un entorno local.

El mayor problema que supone el uso de \texttt{TCP} para nuestros propósitos, es que a nivel de usuario se trabaja como un stream de datos en vez de estar encapsulados en paquetes atómicos, debido a esto, el uso de este protocolo, supone mayor manipulación de la información para poder aislar los distintos mensajes enviados.

Para facilitar estas interacciones, se desarrolla un sistema personalizado que se compone de 2 partes principales:

\subsubsection*{Server-side}
Para mantener la implementación sencilla, el servidor se encarga de levantar un \emph{thread} por cada cliente conectado. Cada uno de ellos sigue la siguiente secuencia:

\begin{enumerate}
	\item Esperar a que haya un \emph{stream} de datos listo para leer o recibir una notificación indicando que el \emph{buffer} de envío (\emph{stream} en cola que se debe enviar al cliente conectado) ha sido escrito por otro \emph{thread}. \textbf{Nota:} ambas cosas pueden pasar simultáneamente.
	\item Recibir los datos desde el cliente remoto (leer datos recibidos) y agregarlos a un \emph{buffer} acumulativo.
\end{enumerate}