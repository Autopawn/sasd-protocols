\documentclass[letterpaper,10pt]{article}

\usepackage[utf8]{inputenc}
\usepackage{fourier}

\usepackage[spanish]{babel} % Language hyphenation and typographical rules

\usepackage{enumitem} % Customized lists
\setlist[itemize]{noitemsep} % Make itemize lists more compact

\usepackage{longtable}

% \usepackage{abstract} % Allows abstract customization
% \renewcommand{\abstractnamefont}{\normalfont\bfseries}
% % ^ Set the "Abstract" text to bold
% \renewcommand{\abstracttextfont}{\normalfont\small\itshape}
% % ^ Set the abstract itself to small italic text

\usepackage{graphicx}

\usepackage{fancyhdr} % Headers and footers
\pagestyle{fancy} % All pages have headers and footers
\fancyhead{} % Blank out the default header
\fancyfoot{} % Blank out the default footer
\fancyhead[C]{Proyecto de Investigación $\bullet$ Agosto 2017}
% ^ Custom header text
\fancyfoot[RO,LE]{\thepage} % Custom footer text

% \usepackage{titling} % Customizing the title section

\usepackage[hidelinks=true]{hyperref} % For hyperlinks in the PDF
\usepackage[font=footnotesize,labelfont=bf]{caption}

\usepackage{amsmath}

\usepackage{geometry}
\geometry{letterpaper}

\title{\Large Proyecto de investigación\\
\huge Análisis de arquitecturas e implementación de protocolos para sincronizar una aplicación interactiva en red
\\ \Large Seminario de Arquitecturas de Sistemas Distribuidos\\
\small U. Técnica Federico Santa María, Departamento de Informática}
% ^ Article title

\author{Francisco Casas Barrientos\\
       \texttt{\small\href{mailto:francisco.casas.13@sansano.usm.cl}{francisco.casas.13@sansano.usm.cl}}
\and
Rodolfo Castillo Mateluna\\
       \small\texttt{\href{mailto:rodolfo.castillo.13@sansano.usm.cl}{rodolfo.castillo.13@sansano.usm.cl}}
\\
}
\date{\today} % Leave empty to omit a date
% \renewcommand{\maketitlehookd}{%
% }

\setlength\parindent{0pt}

\begin{document}

\maketitle

\section*{Introducción}

% El problema adicional en comparación con otros sistemas distribuídos al que se ven vistos los videojuegos es que requieren una interacción muy cercana con el usuario, si se sigue la analogía de una base de datos, un jugador equivaldría a hacer constantemente operaciones de lectura que deben responderse de manera inmediata y como esto no es posible, ...

El objetivo de este trabajo es realizar una comparación práctica de diferentes métodos para lograr consistencia de un estado compartido entre varios clientes en un videojuego, y, posteriormente, realizar una comparación teórica sobre las diferentes arquitecturas que soporten un videojuego masivo multijugador, para evaluar la compatibilidad y desempeño de estas, haciendo un análisis de cuál sería la experiencia del usuario, cómo se manejan las inconsistencias, su seguridad y escalabilidad.

Dentro de este análisis se consideran juegos que requieren rápidos tiempos de respuesta, como son los \emph{shooters}, en desmedro de los que no, por ejemplo, juegos de estrategia basados en turnos, debido a que los primeros traen desafíos mayores también respecto a otros sistemas distribuidos, como la presencia de eventos continuos (que dependen del transcurrir del tiempo) y la constante observación de un usuario al que hay que darle la ilusión de que se encuentra compartiendo el mismo mundo virtual que otros.

Para la comparación práctica de protocolos de sincronización, primero se comenzará listando modelos de consistencia en el Capítulo 1 pues estos sirven para formalizar los requisitos que se esperan de los protocolos, en el Capítulo 2 se listan una serie de protocolos a implementar, en el Capítulo 3 se señalan algunas arquitecturas para videojuegos MMO. En el Capítulo 4, a la mitad del trabajo se discute una implementación en C del \emph{framework} para soportar los diferentes protocolos, así como de un juego que se hizo para realizar pruebas; los resultados cualitativos obtenidos se señalan en el Capítulo 5. Este trabajo finaliza, a partir de la información recopilada y la experiencia obtenida, con tablas que sumarizan conclusiones de los protocolos (Capítulo 6) y arquitecturas (Capítulo 7) y con las conclusiones finales en el Capítulo 8.

% TODO: Indicar separación del trabajo:
% en el capítulo 2 se presentarán los protocolos implementados...
% ^ en el capítulo 4 se explicará más en detalle cómo se implementaron los protocolos y se simularon los problemas de conexión o redes.

% TODO: El trabajo debe enfocarse más en la combinación de las arquitecturas y los protocolos, sobretodo en las conclusiones.

% Señalar el interés: no sólo en esta área pero en varias otras, (sistemas distribuídos, bases de datos distribuídas, sistemas de edición colaborativa) pero la .

% TODO: Recordar diferencia entre métodos sincronicos y asincrónicos señalado en la propuesta de proyecto (todos los objetos avanzan juntos o no), hablar también de simulaciones discretas o continuas.


\section{Modelos de consistencia}

% Explicación de los modelos de consistencia en general.

% Explicar los presentados en los papers vistos.


\section{Protocolos}

% Do not fly too near the sun, Icarus.

% Señalar que como los modelos ultimate consistency están bien estudiados, se usan y se aplican técnicas como dead reckoning para manejar las inconsistencias.

%TODO: Señalar el intento que se hizo de hacer protocolos en que los métodos que no son completamente protocolos sean representativos.

% Señalar algo del protocolo continuous synchronization ?

A continuación se describen los protocolos implementados, cada uno diseñado para ser representativo de una técnica particular para lograr simulaciones distribuidas.

Para todos los protocolos que se revisarán en este trabajo se llegó a la conclusión de que la forma más simple de trabajar es manteniendo simulaciones locales completas del estado del juego (replicas) que son afectadas por eventos de teclado de los clientes, específicamente pulsar y soltar teclas.

El tratar todo el estado del juego como un sólo objeto tiene la desventaja de que todos los nodos deben realizar todo el procesamiento de la partida (por lo tanto se habla de que es un sistema replicado), en contraste con lo que sería que cada nodo compute ciertos objetos y se encargue de informar a otros nodos sobre su posición actualizada (y por ejemplo su velocidad para que puedan aplicar mecanismos básicos de predicción).

A pesar de lo anterior, hacer esto evita tener que programar mecanismos de predicción (pues se utilizan las mismas mecánicas del juego) y tener que aplicar \emph{rollbacks} complejos, e.g. si un nodo envía una actualización errada de un objeto del cual es responsable debido a una interacción con otro objeto que posteriormente descubre que no debería haber pasado, debido a una actualización retrasada de ese objeto; esto obligaría a utilizar antimensajes, por otro lado las pulsaciones de teclas son hechos concretos que no pueden revertirse.

Los protocolos presentados están diseñados para \emph{p2p}, salvo el basado en \emph{dead-reckoning} que corresponde a una arquitectura \emph{cliente-servidor}.

\subsection{Ejecución}

Un evento consistirá entonces en si se pulsó o el soltó una tecla, la tecla y el jugador que pulsó dicha tecla. Un evento tendrá asociada una \emph{frame} de ejecución programada (la cual es diferente a la \emph{frame} en que se pulsó la tecla, debido al \emph{local-lag}).

En términos generales, la ejecución se concretiza por una función determinista:
\begin{align*}
    f : S \times \mathbb{P}(E) \rightarrow S
\end{align*}
Que recibe un estado actual y un conjunto de eventos (cuyo tiempo de ejecución corresponde al del estado actual) y entrega el estado siguiente. Donde $S$ es el conjunto de posibles estados y $E$ el conjunto de posibles eventos.

\subsection{Detalles de los protocolos}

% TODO: Agregar dibujos?

\subsubsection{Protocolo centralizado con \emph{dead-reckoning}}

En este protocolo el servidor enviará constantemente los estados completos del juego a los clientes, cada vez que llegue un estado del juego a un cliente, llegará en el pasado producto del retraso de la red, así que este deberá avanzarlo con la función hasta alcanzar el estado que se debería mostrar actualmente.

Los clientes enviarán sus eventos al servidor (el cuál también estará corriendo una simulación) y este los ejecutará a penas lleguen, los otros clientes verán los efectos de este evento cuando les lleguen los nuevos estados del servidor, % inevitablemente observando saltos
el cliente podrá registrar eventos propios en su simulación local intentando ajustar el \emph{local-lag} para que su \emph{frame} de ejecución sea la misma que la en que el evento llegue al servidor.

\subsubsection{Protocolo con Time-warp}

% TODO: señalar lo que se pueda sacar del paper A Testbed for P2P gamming using Time-warp.

Este protocolo está diseñado para arquitecturas \emph{p2p} más que para \emph{cliente-servidor}, e inicialmente fue presentado como una alternativa a los otros protocolos basados en \emph{Network Paradigm} para sistemas no replicados sino distribuidos, cada cliente envía sus eventos a todos los otros clientes, agregando un \emph{local-lag} que estime conveniente, uno muy bajo generará varias inconsistencias, requiriendo muchos \emph{rollbacks} en otros clientes, mientras que uno muy alto hará el juego notoriamente poco responsivo al jugador.

Cada cliente almacena los estados pasados en una \emph{traza} de estados y también los eventos que le llegan, si llega un evento con \emph{frame} de ejecución en el pasado, debe descartar los estados de la \emph{traza} desde ese punto en adelante y recomputarla hasta el estado actual.

\subsubsection{Protocolo con Fast event ordering}

Este protocolo se basa en mantener los clientes con diferentes desfasados entre sí, aplicando un \emph{local-lag} grande a los eventos propios y un ajuste a los eventos que llegan de otros clientes de manera que se ejecuten en el momento correcto considerando el desfase propio (se logre la \emph{perceptive consistency}).

La idea principal es conocer los retrasos entre cada par de clientes y aplicar retrasos suficientemente grandes para que, suponiendo de manera optimista que estos retrasos no aumenten con el tiempo, no sea necesario aplicar \emph{rollbacks}, los retrasos se minimizan a costa de que los nodos queden desfasados entre sí.

El protocolo que se implementará manejará las inconsistencias de la misma manera que \emph{time-warp}.

% Proveé un estado consistente tan rápido como lo permite la red y trabaja bien con estrategias basadas en \emph{dead reackoning}, funcionando con media discreta y continua.

\subsubsection{Protocolo con Trailing State Syncronization}

Diseñado para los requerimientos de shooter first-person distribuidos. Mantiene en memoria estados correspondientes a ciertas cantidades de \emph{frames} en el pasado (ordenados según esta cantidad de \emph{frames}) llamados \emph{trailing states}, estos se avanzan junto con el estado que se está mostrando al jugador, de manera que su retraso con la \emph{frame} actual se mantiene constante. A diferencia de \emph{time-warp}, se guardan menos estados.

Cuando se detecta una inconsistencia, se toma el \emph{trailing state} de la \emph{frame} más nueva que no la tiene y se hace avanzar hasta obtener la \emph{frame} actual, dejando copias que reemplazan los \emph{trailing states} correspondientes a las \emph{frames} más actuales a la inconsistencia. Se requiere que el último \emph{trailing state} siempre sea más antiguo que cualquier inconsistencia.

Aunque la versión que se programará trata eventos pasados como inconsistencias, cave mencionar que TSS en su forma original detecta las inconsistencias cuando los cambios producidos por un evento son diferentes a los cambios producidos por el mismo evento ejecutado en los \emph{trailing states} adelantados. Esta comparación, que se realiza cuando el \emph{trailing state} anterior alcanza la \emph{frame} del evento, evita tener que hacer \emph{rollback} cuando los eventos no generaron cambios significativos, además el ``esperar'' hasta llegar a dicha \emph{frame} permite que varias inconsistencias se manejen en un mismo \emph{rollback} disminuyendo el procesamiento requerido.

% \subsubsection{Otros protocolos y métodos}

% NOTE: Se menciona bucket syncronization, breathing bucket syncronization y  time-warp syncronization, que fueron creador primero para simulaciones militares.

% Continuous synchronization, Bucket synchronization, Predictive Time management.


\section{Arquitecturas}

En la medida que se intenta mantener una realidad compartida entre cada vez más clientes, la segunda parte del \emph{state melding}, la diseminación de las actualizaciones se hace cada vez más importante debido a las limitaciones en la capacidad de procesamiento y la red. A continuación se nombran algunos modelos de arquitecturas que enfrentan este problema de escalabilidad.

\subsection{Detalles de las arquitecturas}

\subsubsection{Simulator-Centric}

Utilizado por \emph{Second-Life} y \emph{Eve Online}, en que un servidor tiene autoridad sobre un área del mundo central.

\subsubsection{Darkstar}

\subsubsection{Peer-to-peer}

\subsubsection{Mirrored game architecture}

% Comentado por Cronin, Kurc, Filstrup y Jamin (for mirrored game architectures).


\section{Implementación}

\subsection{Videojuego}

El videojuego consiste en una serie de cuadrados \emph{negras} y de colores, las primeras se mueven automáticamente producto de reglas físicas previamente determinadas, mientras que las segundas son manejadas por los diferentes clientes (mediante el control de su aceleración), se tienen las siguiente reglas:

\begin{itemize}
\item Si dos pelotas de colores colisionan, la que lleva menos velocidad se destruye (luego vuelve a aparecer en otro lugar del mapa) y el jugador que maneja la otra gana un punto.
\item Si en una colisión al menos una de las pelotas es negra, ambas rebotan.
\item Si una pelota sale del mapa, es destruida (volviendo a aparecer en otro lado), si es de color el jugador que la maneja pierde un punto.
\end{itemize}

\subsection{Estructura general del \emph{framework}}

La implementación se divide en una sección de manipulación de \emph{streams} por la red, otra que actualiza los estados de la aplicación en respuesta a eventos y una tercera que se encarga de renderizar el estado de la aplicación.

Esta elección de estructura se basa en la modularidad de cada funcionalidad, ya que el manejo de la red se puede reutilizar para distintos protocolos y la lógica del cliente es completamente atómica y \textbf{determinista}.

El determinismo es clave para la implementación del \emph{framework}, ya que la sincronización no se basa a partir de paquetes que definen el estado en los distintos clientes, si no que los eventos emitidos en cada uno de ellos. De esta manera, si la lógica de la aplicación es determinista, basta con intercambiar el input percibido en cada cliente con el resto de los nodos para poder implementar la sincronización dependiendo del protocolo escogido.

Como detalles adicionales, cabe destacar el desarrollo de diferentes utilidades que ofrecen un uso más sencillo de estructuras de datos dinámicas y serialización de datos. Además en la sección de renderización del cliente, el dibujo de cada iteración no es borrado completamente de la pantalla para poder ver claramente la consistencia que otorga el protocolo.\footnote{https://github.com/Autopawn/sasd-protocols}

\subsection{Simulación de la latencia entre clientes}

Para hacer comparaciones entre las distintas implementaciones de protocolos se desarrolla un método de simulación de latencia entre clientes. Para mantener la simulación simple, cada cliente es inicializado con una matriz de latencias con dimensiones $n\times n$ donde $n$ es la cantidad máxima de jugadores. Como su nombre lo dice, esta matriz relaciona la latencia \textbf{dirigida} entre 2 jugadores.

Para comprender su funcionamiento, se deben entender los conceptos:
\begin{itemize}
	\item \emph{frame}: se trata del número de iteración del procesamiento en el lado del cliente en el que cierto evento debe ser ejecutado.
	\item \emph{summoned\_frame}: se trata del número de iteración del procesamiento en el lado del cliente en el que cierto evento fue emitido.
\end{itemize}

Entonces, si no existe latencia simulada entre 2 clientes, un paquete es puesto en cola para ser procesado en la iteración \emph{frame} inmediatamente al ser recibido, mientras que, si existe una latencia simulada de \emph{lat} iteraciones, ese mismo paquete será retenido, y se pondrá en cola para la iteración \emph{frame} recién cuando el cliente esté en la iteración \emph{summoned\_frame} + \emph{lat}.

% Descripción del juego de pruebas, extraer parte importante del informe de avance.

% - Implementación de los protocolos y cómo se diseñó un sistema general para tratar con todos ellos.
% - Implementación de las simulaciones de latencia, pérdida de paquetes y jitter.
% - Setup de los nodos en la red.

% Señalar que se requerián todos los clientes conectados al principio para simplificar.

% Señalar que vamos a abstraer a los clientes de lo que hace el servidor, salvo por el hecho de que el cliente sabe que a través del servidor se hará el broadcast.

% La programación del juego está dada por una función determinista f(s_i,e) = s_{i+1}, donde s_i es el estado en una frame específica y e es un conjunto de eventos realizados por usuarios.

\subsection{Networking engine}
Para la implementación de las interacciones de red, se usa como base el protocolo \texttt{TCP}. Esta elección se sustenta en que no es necesario tomar en cuenta el orden en que se envían los mensajes y su integridad; si bien es cierto que existen desventajas respecto a la latencia por el \emph{overhead} que produce, no causan mayor impacto en un entorno local.

El mayor problema que supone el uso de \texttt{TCP} para nuestros propósitos, es que a nivel de usuario se trabaja como un stream de datos en vez de estar encapsulados en paquetes atómicos, debido a esto, el uso de este protocolo, supone mayor manipulación de la información para poder aislar los distintos mensajes enviados.

Para facilitar estas interacciones, se desarrolla un sistema personalizado que se compone de 2 partes principales:

\subsubsection*{Server-side}
Para mantener la implementación sencilla, el servidor se encarga de levantar un \emph{thread} por cada cliente conectado. Cada uno de ellos sigue la siguiente secuencia:

\begin{enumerate}
	\item Esperar a que haya un \emph{stream} de datos listo para leer o recibir una notificación indicando que el \emph{buffer} de envío (\emph{stream} en cola que se debe enviar al cliente conectado) ha sido escrito por otro \emph{thread}. \textbf{Nota:} ambas cosas pueden pasar simultáneamente.
	\item Recibir los datos desde el cliente remoto (si es que hay información disponible para leer) y agregarlos a un \emph{buffer} acumulativo.
	\item Si hay suficiente información en el \emph{buffer} acumulativo, seccionarlo tantas veces como sea posible en paquetes delimitados.
	\item Escribir todos los paquetes seccionados a un \emph{buffer} histórico. Este buffer mantiene el stream que ha enviado el cliente durante todo el tiempo que ha estado conectado.
	\item Escribir todos los paquetes seccionados en el \emph{buffer} de envío de todos los otros clientes conectados.
	\item Enviar toda la información acumulada en el \emph{buffer} de envío.
\end{enumerate}

% \item Recibir los datos desde el cliente remoto (leer datos recibidos) y agregarlos a un \emph{buffer} acumulativo.
% \end{enumerate}

Con estos sencillos pasos se puede lograr un sistema que sea capaz de admitir conexiones en distintos puntos en la línea temporal manteniendo a todos los clientes sincronizados.

\subsubsection*{Client-side}

Para que las operaciones de red no interfieran con la lógica del cliente (que actualiza y renderiza el estado del juego), se genera una librería que hace la tarea de seccionar un \emph{stream} entrante en paquetes del protocolo, en segundo plano, de modo que se pueda obtener una cola de paquetes recibidos asíncronamente en la fase de procesamiento de eventos.


\section{Resultados}

% Identificar cualitativamente los resultados del retraso de la red en cada uno de los métodos.


\section{Comparación de protocolos}

% Tabla comparativa de protocolos, sobre:
% - Consistencia y responsividad
% - Seguridad
% - Escalabilidad
%
% Clasificar según 3° paper del informe de avance.

\small
\begin{longtable}{|p{2cm}|p{4cm}|p{4cm}|p{4cm}|}
\hline
\textbf{Protocolo}
&
\textbf{Consistencia vs responsividad}
&
\textbf{Seguridad}
&
\textbf{Escalabilidad}
\\ \hline
Centralizado con \emph{dead-reckoning}
&
El retraso aumenta dado que los eventos tienen que ser procesados por el servidor, se puede aplicar un \emph{local-lag} una vez los eventos llegan al servidor para disminuir las inconsistencias en los clientes, pero a costa de una responsividad aun peor.
&
Que los eventos se ejecuten al momento que lleguen al servidor y que este mantenga una copia autoritaria del estado del juego asegura que los clientes no abusen con los tiempos de ejecución asociados a sus eventos.
&
El servidor se convierte en un cuello de botella, por el costo de procesamiento de correr una instancia del juego y por la capacidad de la red, ya que debe conectarse con todos los clientes.
\\ \hline
T
&
Los \emph{local-lags} se pueden ajustar para balancear de la mejor manera posible los efectos del retraso de la red entre consistencia y responsividad.
&
Por si sólo, no hay forma de asegurar que los clientes asocien tiempos de ejecución correctos a sus eventos, pudiendo lograr una mayor responsividad afectando la consistencia del resto.
&
A parte de tener que hacer la actualización de estado determinista, abstrae la programación del juego. \newline
Se puede extender a varios clientes sin modificar el código, pero por ser \emph{p2p} está limitado por la cantidad de conexiones, requiriendo un método para estructurar la red más complejo que conectar todos con todos. El mantener y copiar constantemente estados anteriores y los \emph{rollbacks} requieren varias veces más capacidad de procesamiento y memoria por parte de los clientes.
\\ \hline
Fast event ordering
&
Los retrasos de la red afectan completamente responsividad en pos de lograr consistencia total.
&
Conociendo cómo funciona el algoritmo, un cliente puede mentir respecto a sus retrasos con los otros clientes para beneficiarse o afectar el correcto funcionamiento de la red, al igual que los otros protocolos \emph{p2p} puede enviar eventos diferentes a clientes diferentes para hacer diverger las replicas.
&
Al igual que \emph{Timewarp} está limitado por la cantidad de conexiones, si se busca una mayor cantidad de clientes y que estos no esté conectados todos con todos se requiere un método para hacer una red estructurada.
\newline Consume muchos menos recursos de parte del cliente.
\\ \hline
Trailing State Syncronization
&
La corrección de inconsistencias no es inmediata como en \emph{timewarp}, se espera hasta que se detectan, esto se suma al retraso de la red, por otro lado, si los eventos no afectan el estado del resultado, no se requiere hacer \emph{rollback}.
&
Tiene los mismos problemas que \emph{Timewarp}.
&
Que se espere hasta detectar la inconsistencia permite resolver varias inconsistencias al mismo tiempo (aunque afecta un poco la responsividad), ahorrando procesamiento, al mismo tiempo, se requiere mucho menos memoria para guardar los estados anteriores, aunque se debe avanzar todos los \emph{trailing states} cada \emph{frame}, lo que puede ser desventajoso si esto es mucho más costoso que copiar un estado.
\\ \hline
\end{longtable}
\normalsize

% TODO: Conclusiones (leer paper de seguridad).
% En general puede ser malo revelar todo el estado al jugador, si se ocultan cosas al nivel del displayer del cliente pueden hacerse injusticias...

% TODO: Se destaca que una característica deseable puede ser el que los clientes afectados por inconsistencias o mala responsibidad sean los que tienen peor red respecto al resto.

% TODO: Explicar un poco como cambiarán las cosas si se distribuyera el procesamiento de los estados.

% TODO: Señalar los protocolos p2p y sus problemas (enviar eventos diferentes a otros clientes para hacer diverger el estado del juego o enviar eventos en el pasado) el problema de llegar a un consenso o excluír al jugador con faltas.

% NOTE: Señalar que aquí se pondrán conclusiones también en la introducción.


\section{Comparación de arquitecturas}

% Tabla comparativa de las arquitecturas, sobre:
% - Consistencia y responsividad
% - Seguridad
% - Escalabilidad
%
% Clasificar según 5° paper del informe de avance.





\small
\begin{longtable}{|p{2cm}|p{4cm}|p{4cm}|p{4cm}|}
\hline
\textbf{Arquitectura} & \textbf{Consistencia vs responsividad} & \textbf{Seguridad} & \textbf{Escalabilidad} \\ \hline
Simulator-Centric &
consistencia y responsividad &
seguridad &
escalabilidad \\
\hline
Darkstar &
consistencia y responsividad &
seguridad &
escalabilidad \\
\hline
Peer-to-peer con router (Croquet) &
consistencia y responsividad &
seguridad &
escalabilidad \\
\hline
Mirrored game architectures &
consistencia y responsividad &
seguridad &
escalabilidad \\
\hline
\end{longtable}
\normalsize


\section{Conclusiones}

Respecto a las arquitecturas, \emph{peer-to-peer} con \emph{router} parece ser el más escalable de todos, quedó claro que el las arquitecturas \emph{simulator-centric} tienen un límite, aunque si en el mundo virtual que se quiere lograr es factible crear islas de objetos (y jugadores) o lograr un buen \emph{interest management} que \emph{granule} de manera abrupta los datos y el procesamiento, por razones de simpleza conviene usar \emph{simulator-centric}. En general los protocolos \emph{peer-to-peer} enfrentan desafíos de seguridad mucho mayores pues tienen que confiar en los clientes, pero el aplicar un \emph{router} resuelve este problema de seguridad y también el de a consistencia.

En los \emph{papers} investigados, no se encontraron sugerencias de métodos de interpolación en la visualización que permitan mostrar al jugador las inconsistencias de manera más suavizada, sin embargo parece lógico que esto debería realizarse (aunque los nuevos estados computados no se muestren de manera inmediata). En \cite{li2004supporting} las inconsistencias se muestran interpoladas aunque esto es el resultado del protocolo más que una técnica separada. Otro técnica paralela que podría ser necesario de usar, dependiendo del tipo de simulación, es separar los cambios \emph{duros} de los \emph{suaves}, siendo los primeros cosas significativas para el jugador, como la muerte de su avatar o el cambio en el puntaje, estos cambios no deberían revertirse por lo que, dentro de lo posible, sólo deberían notificarse cuando ha pasado un tiempo de confirmación.

Se concluye que los protocolos se puede adaptar a casi cualquier arquitectura, pues se encargan de tratar la comunicación entre cada par de clientes y el manejo de la información local, mientras que el de las arquitecturas se preocupan de qué nodos conectar y que infraestructura de apoyo brindar, por ejemplo el protocolo con \emph{dead-reckoning} se puede implementar en una arquitectura \emph{peer-to-peer} si uno de los clientes actúa como servidor. Aun así, hay que considerar de la combinación que se está ocupando si esto refuerza los cuellos de botella de procesamiento y uso de la red, así como tener las consideraciones de seguridad correspondientes, intentado evitar responsabilizar a los clientes de otra cosa que sus eventos, de otra manera habría que añadir mecanismos de detección de trampas. De hecho, arquitecturas más avanzadas usan diferentes protocolos en conjunto para conjuntos de objetos o eventos que tienen requisitos diferentes.

% Aunque el trabajo original indica que se puede usar \emph{dead-reckoning} para hacer aparentemente más responsivo la simulación, esto no es posible sin lograr inconsistencias a nivel local, por lo que se requeriría, por ejemplo, algún método de interpolación para \emph{suavizar} la percepción de estas inconsistencias -> ojo, pero esto rompería el que se cumplan las reglas físicas en ciertos casos de mucha discrepancia. señalar que \cite{li2004supporting} hace esto (aunque no se programó).


% Concluír cosas...
%
% E.g.:
% - Qué protocolos son más convenientes para qué arquitecturas.
% - Qué protocolos y arquitecturas son mejores para qué tipo de juego (considerar por ejempo la cantidad de jugadores, la reactividad necesaria, cómputo necesario, consistencia, etc.).
% - Qué tecnologías se pueden utilizar para prevenir problemas de seguridad de los protocolos que los tienen.
% - ¿Algunas conclusiones sobre descentralización?
% - Qué cosas se podrían investigar más.
% - ... ¡Y muchas más!



% Hay una diferencia entre hacer dead reckoning con objetos manejados por cada usuario a hacerlo con eventos... elaborate more.

% Local lag en el cliente, local lag en el servidor, servidor resuelve el problema de seguridad de que los clientes pueden mentir respecto al local lag.

% Eventos "suaves" y eventos "duros", ¿En qué paper leí eso? ¿El de TSS?

% Aunque el trabajo original indica que se puede usar \emph{dead-reckoning} para hacer aparentemente más responsivo la simulación, esto no es posible sin lograr inconsistencias a nivel local, por lo que se requeriría, por ejemplo, algún método de interpolación para \emph{suavizar} la percepción de estas inconsistencias -> ojo, pero esto rompería el que se cumplan las reglas físicas en ciertos casos de mucha discrepancia. señalar que \cite{li2004supporting} hace esto (aunque no se programó).

% TODO: Agregar y revisitar conclusiones del informe de avance.

% El que la actualización de los estados respecto al tiempo sea determinsta es otra forma de evitar la necesidad de actualizaciones constantes y permitir otros modelos de consistencia, ¿Se convierte en una necesidad? <- Resolver esto.
% El problema es la lectura constante de la pantalla por parte del usuario final.

% Fenómenos de seguridad que pasan ante un usuario con percepción limitada, cómo se puede aprovechar.

% Importancia de los cambios "importantes", no se trató, el identificarlos, delay de confirmación, muchas otras cosas... (separar cambios importantes de cambios suceptibles a reparación).

% Temas de justicia, ¿Penalizar peores conexiones a la red?

% En los juegos coolaborativos se puede tolerar más las inconsistencias.

% Sistema híbrido no determinista en que haya comunicación p2p y un servidor con estados mandatorios para coordinar cada cierto tiempo.

% Arquitecturas más avanzadas combinan los métodos y protocolos, categorizando eventos según sus propiedades.

% Conclusion sobre (me olvidé)


% TODO: Agregar referencias.

\bibliographystyle{plain}
\bibliography{main}{}

\end{document}
