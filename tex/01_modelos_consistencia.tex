\section{Modelos de consistencia}

% Explicación de los modelos de consistencia en general.

El principal problema a resolver en una simulación o aplicación distribuida es la \textbf{consistencia}, que se entiende cómo la capacidad de encontrar el orden correcto de las operaciones que dependen entre sí.

Entre los métodos que han surgido para resolver esta problemática, es emblemático el de los relojes lógicos de L. Lamport \cite{lamport1978time}. Éste último y sus derivados están enfocados en problemas \textbf{discretos}, garantizando que la secuencia correcta de eventos haya sido propagada entre los participantes resultando en un estado coherente entre ellos. Sin embargo, los videojuegos se tratan de problemas \textbf{continuos} en que los cambios de estado no dependen sólo de eventos sino también del transcurrir del tiempo,
no siendo una opción detener la ejecución a la espera de un evento, ya que se requiere que los resultados se vean reflejados como si los eventos hubieran sido ejecutadas en el punto que les corresponde en la línea temporal\cite{mauve2004local}.

Otra requerimiento de los videojuegos es la \textbf{responsividad}, que consiste en que el jugador sienta que sus acciones tienen efecto inmediato en el mundo del juego, osea que el tiempo de ejecución de sus eventos sean cercanos al momento en que son emitidos. Ante los retrasos de la red, la consistencia y la responsividad son objetivos encontrados, hacer que los retrasos no afecten a uno implica que tengan que afectar al otro.

% Explicar los presentados en los papers vistos.

% TODO: Explicar más modelos de consistencia faltantes.

\subsection{Continuous consistency}

% TODO: Releer paper y colocar aquí.
% NOTE: Colocarlo antes o después de perceptive coonsistency?

\subsection{Perceptive consistency}

Señalada en \cite{bouillot2005fast}, se basa en que se cumplan dos propiedades, la primera es $\Delta$ \emph{legality} y la otra es \emph{simultaneity}.

La $\Delta$ \emph{legality} se puede entender como que la diferencia de tiempo entre que se ejecuta cualquier par de eventos en cualquier cliente que los recibe es la misma que la diferencia de tiempo entre que se emiten en su origen. Mientras que la \emph{simultaneity} se puede entender como que la diferencia de tiempo entre que se ejecuta cualquier par de eventos que se tome es la misma para todos los clientes.

Este tipo de consistencia equivaldría, en el contexto multimedia, a que todos los clientes vean una misma película, permitiendo que algunos la vean desfasados respecto a otros.

% Todo esto tiene que ver con \cite{bouillot2005fast}


% Time-Sensitive Distributed Multiplayer Games son los que requieren que el usuario interactúe con su propia "réplica" local del juego (responsividad máxima), cada réplica ejecuta las actualizaciones en el mismo órden.

% El tiempo físico entre la ejecución de dos actualizaciones es el mismo para todos los usuarios, (simultaneity si las actualizaciones vienen de 2 fuentes diferetes y \Delta legality si vienen del mismo source.

% Dead Reckoning se basa en recibir los objetos de otro host y computar el estado actual que debería tener de acuerdo a la información de avance (posición,velocidad,etc.),
