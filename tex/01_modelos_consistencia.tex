\section{Modelos de consistencia}

% Explicación de los modelos de consistencia en general.

% Explicar los presentados en los papers vistos.

\subsection{Perceptive consistency}

Señalada en \cite{bouillot2005fast}, se basa en que se cumplan dos propiedades, la primera es $\Delta$ \emph{legality} y la otra es \emph{simultaneity}.

La $\Delta$ \emph{legality} se puede entender como que la diferencia de tiempo entre que se ejecuta cualquier par de eventos en cualquier cliente que los recibe es la misma que la diferencia de tiempo entre que se emiten en su origen. Mientras que la \emph{simultaneity} se puede entender como que la diferencia de tiempo entre que se ejecuta cualquier par de eventos que se tome es la misma para todos los clientes.

Este tipo de consistencia equivaldría, en el contexto multimedia, a que todos los clientes vean una misma película, permitiendo que algunos la vean desfasados respecto a otros.

% Todo esto tiene que ver con \cite{bouillot2005fast}


% Time-Sensitive Distributed Multiplayer Games son los que requieren que el usuario interactúe con su propia "réplica" local del juego (responsividad máxima), cada réplica ejecuta las actualizaciones en el mismo órden.

% El tiempo físico entre la ejecución de dos actualizaciones es el mismo para todos los usuarios, (simultaneity si las actualizaciones vienen de 2 fuentes diferetes y \Delta legality si vienen del mismo source.

% Dead Reckoning se basa en recibir los objetos de otro host y computar el estado actual que debería tener de acuerdo a la información de avance (posición,velocidad,etc.),
