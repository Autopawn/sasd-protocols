\section{Protocolos}

% Do not fly too near the sun, Icarus.

% Señalar que como los modelos ultimate consistency están bien estudiados, se usan y se aplican técnicas como dead reckoning para manejar las inconsistencias.

%TODO: Señalar el intento que se hizo de hacer protocolos en que los métodos que no son completamente protocolos sean representativos.

% Señalar algo del protocolo continuous synchronization ?

A continuación se describen los protocolos implementados, cada uno diseñado para ser representativo de una técnica particular para lograr simulaciones distribuídas.

Para todos los protocolos que se revisarán en este trabajo se llegó a la conclusión de que la forma más simple de trabajar es manteniendo simulaciones locales completas del estado del juego (replicas) que son afectadas por eventos de teclado de los clientes, específicamente pulsar y soltar teclas.

El tratar todo el estado del juego como un sólo objeto tiene la desventaja de que todos los nodos deben realizar todo el procesamiento de la partida (por lo tanto se habla de que es un sistema replicado), en contraste con lo que sería que cada nodo compute ciertos objetos y se encargue de informar a otros nodos sobre su posición actualizada (y por ejemplo su velocidad para que puedan aplicar mecanismos básicos de predicción).

A pesar de lo anterior, hacer esto evita tener que programar mecanismos de predicción (pues se utilizan las mismas mecánicas del juego) y tener que aplicar \emph{rollbacks} complejos, e.g. si un nodo envía una actualización errada de un objeto del cual es responsable debido a una interacción con otro objeto que posteriormente descubre que no debería haber pasado, debido a una actualización retrasada de ese objeto; esto obligaría a utilizar antimensajes, por otro lado las pulsaciones de teclas son hechos concretos que no pueden revertirse.

Los protocolos presentados están diseñados para \emph{p2p}, salvo el basado en \emph{dead-reckoning} que corresponde a una arquitectura \emph{cliente-servidor}.

\subsection{Ejecución}

Un evento consistirá entonces en si se pulsó o el soltó una tecla, la tecla y el jugador que pulsó dicha tecla. Un evento tendrá asociada una \emph{frame} de ejecución programada (la cual es diferente a la \emph{frame} en que se pulsó la tecla, debido al \emph{local-lag}).

En términos generales, la ejecución se concretiza por una función determinista:
\begin{align*}
    f : S \times \mathbb{P}(E) \rightarrow S
\end{align*}
Que recibe un estado actual y un conjunto de eventos (cuyo tiempo de ejecución corresponde al del estado actual) y entrega el estado siguiente. Donde $S$ es el conjunto de posibles estados y $E$ el conjunto de posibles eventos.

\subsection{Detalles de los protocolos}

% TODO: Agregar dibujos?

\subsubsection{Protocolo centralizado con \emph{dead-reckoning}}

En este protocolo el servidor enviará constantemente los estados completos del juego a los clientes, cada vez que llegue un estado del juego a un cliente, llegará en el pasado producto del retraso de la red, así que este deberá avanzarlo con la función hasta alcanzar el estado que se debería mostrar actualmente.

Los clientes enviarán sus eventos al servidor (el cuál también estará corriendo una simulación) y este los ejecutará a penas lleguen, los otros clientes verán los efectos de este evento cuando les lleguen los nuevos estados del servidor, % inevitablemente observando saltos
el cliente podrá registrar eventos propios en su simulación local intentando ajustar el \emph{local-lag} para que su \emph{frame} de ejecución sea la misma que la en que el evento llege al servidor.

\subsubsection{Protocolo con Time-warp}

% TODO: señalar lo que se pueda sacar del paper A Testbed for P2P gamming using Time-warp.

Este protocolo está diseñado para arquitecturas \emph{p2p} más que para \emph{cliente-servidor}, e inicialmente fue presentado como una alternativa a los otros protocolos basados en \emph{Network Paradigm} para sistemas distribuídos y no replicados sino distribupidos, cada cliente envía sus eventos a todos los otros clientes, agregando un \emph{local-lag} que estime conveniente, uno muy bajo generará varias inconsistencias, requiriendo muchos \emph{rollbacks} en otros clientes, mientras que uno muy alto hará el juego notoriamente poco responsivo al jugador.

Cada cliente almacena los estados pasados en una \emph{traza} de estados y también los eventos que le llegan, si llega un evento con \emph{frame} de ejecución en el pasado, debe descartar los estados de la \emph{traza} desde ese punto en adelante y recomputarla hasta el estado actual.

\subsubsection{Protocolo con Fast event ordering}

Este protocolo se basa en mantener los clientes con diferentes desfasados entre sí, aplicando un \emph{local-lag} grande a los eventos propios y un ajuste a los eventos que llegan de otros clientes de manera que se ejecuten en el momento correcto considerando el desfase propio (se logre la \emph{perceptive consistency}).

La idea principal es conocer los retrasos entre cada par de clientes y aplicar retrasos suficientemente grandes para que, suponiendo de manera optimista que estos retrasos no aumenten con el tiempo, no sea necesario aplicar \emph{rollbacks}, los retrasos se minimizan a costa de que los nodos queden desfasados entre sí.

El protocolo que se implementará manejará las inconsistencias de la misma manera que \emph{time-warp}.

% Proveé un estado consistente tan rápido como lo permite la red y trabaja bien con estrategias basadas en \emph{dead reackoning}, funcionando con media discreta y continua.

\subsubsection{Protocolo con Trailing State Syncronization}

Diseñado para los requerimientos de shooter first-person distribuídos. Mantiene en memoria estados correspondientes a ciertas cantidades de frames en el pasado (ordenados según esta cantidad de frames) llamados \emph{trailing states}, estos se avanzan junto con el estado que se está mostrando al jugador, de manera que su retraso con la frame actual se mantiene constante. A diferencia de \emph{timewarp}, se guardan menos estados.

Cuando se detecta una inconsistencia, se toma el \emph{trailing state} de la frame más nueva que no la tiene y se hace avanzar hasta obtener la frame actual, dejando copias que reemplazan los \emph{trailing states} correspondientes a las frames más actuales a la inconsistencia. Se requiere que el útimo \emph{trailing state} siempre sea más antiguo que cualquier inconsistencia.

Aunque la versión que se programará trata eventos pasados como inconsistencias, cave mencionar que TSS en su forma original detecta las inconsistencias cuando los cambios producidos por un evento son diferentes a los cambios producidos por el mismo evento ejecutado en los \emph{trailing states} adelantados. Esta comparación, que se realiza cuando el \emph{trailing state} anterior alcanza la \emph{frame} del evento, evita tener que hacer \emph{rollback} cuando los eventos no generaron cambios significativos, además el ``esperar'' hasta llegar a dicha \emph{frame} permite que varias inconsistencias se manejen en un mismo \emph{rollback} disminuyendo el procesamiento requerido.

% \subsubsection{Otros protocolos y métodos}

% NOTE: Se menciona bucket syncronization, breathing bucket syncronization y  time-warp syncronization, que fueron creador primero para simulaciones militares.

% Continuous synchronization, Bucket synchronization, Predictive Time management.
